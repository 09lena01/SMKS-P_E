\documentclass[11pt]{scrreport}
\usepackage[utf8]{inputenc}
\usepackage{lmodern}
\usepackage[ngerman] {babel}
\usepackage[T1]{fontenc}
\usepackage{graphicx,float}
\usepackage{booktabs}
\usepackage{amsfonts}
\usepackage{amsmath}
\usepackage[dvipsnames]{xcolor}
\usepackage{suffix}
\usepackage{xstring}
\usepackage[onehalfspacing]{setspace}
\usepackage{chemformula}
\usepackage[breaklinks,pdfpagelabels,hidelinks]{hyperref}%
\usepackage[figure]{hypcap}
\usepackage{cleveref}
\usepackage[style=chem-angew,doi,chaptertitle,articletitle]{biblatex}
\usepackage{chemfig}
\usepackage[right=2.5cm, left=2.5cm, top=2.5cm, bottom=2.5cm]{geometry}%,showframe
\usepackage{ragged2e}
\usepackage{caption}
\usepackage[section]{placeins}
\usepackage{colortbl}
\usepackage{float}
\usepackage{mathtools}
\usepackage{multirow}
\usepackage{siunitx}
\sisetup{detect-all, locale = DE}
\usepackage[useregional]{datetime2}
\usepackage[nopostdot,style=super,acronyms,nonumberlist,toc]{glossaries-extra}
\usepackage{tikz}
\usepackage{datatool}
\usepackage{csquotes}
\usepackage[list=true]{subcaption}
\setcounter{tocdepth}{4}
\setcounter{secnumdepth}{4}
\renewcommand{\familydefault}{\sfdefault}
\addbibresource{FRET.bib}
\let\cite=\supercite
\numberwithin{equation}{chapter}
\addto{\captionsgerman}{\renewcommand*{\contentsname}{Inhaltsverzeichnis}}


\DefineBibliographyExtras{german}{%
	\DeclareBibstringSet{latin}{andothers,ibidem}%
	\DeclareBibstringSetFormat{latin}{\mkbibemph{#1}}%
}
\UndefineBibliographyExtras{german}{%
	\UndeclareBibstringSet{latin}%
}
\DefineBibliographyStrings{german}{%
	andothers = {et al\adddot},
}
\renewcommand\mkbibnamefamily[1]{\textsc{#1}}
\raggedbottom
\begin{document}
	\begin{titlepage}
		\setcounter{page}{0}
		\centering
		\includegraphics[width=0.6\textwidth]{"HHU_Logo_WortBildMarke.png"} 
		\vfill
		{\LARGE\bfseries Protokoll\\Förster Resonanz Energie Transfer (FRET) an markierter DNA -- E} \\
		\vfill
		Mathematisch-Naturwissenschaftliche Fakultät\\
		Heinrich Heine-Universität Düsseldorf\\
		\vfill
		für das Modul\\
		Pflichtpraktikum Physikalische Chemie (SMKS-P)\\
		im Wintersemester 2025/26
		\vfill
		Betreuende:r Assistent:in: Paul Lauterjung\\
		Abgabedatum: \today
		\vfill
		von:\\
		Lena-Marie Aßmann  \\
		\href{mailto:lena-marie.assmann@hhu.de}{lena-marie.assmann@hhu.de}\\
		Matrikelnr.: 3121504\\
		\vfill
	\end{titlepage}
	\pagenumbering{Roman}
	\setcounter{page}{1}
	\tableofcontents
	\newpage
	\pagenumbering{arabic}
	\chapter{Einleitung}
	In diesem Versuch wird mithilfe von DNA-Strängen der Abstand \(R_{DA}\) zwischen zwei Basenpaaren bestimmt. Definierten Positionen der DNA-Stränge wurden mit den Fluorophoren Alexa488 (Donor) und Cy5 (Akzeptor) markiert. Hierbei wird der Donorfarbstoff optisch angeregt und die entstehende Fluoreszenz wird detektiert. Ein Teil der Anregungsenergie des Donors wird über den Förster-Resonanz-Energietransfer (FRET) strahlungslos auf den Akzeptor übertragen, wodurch dieser ebenfalls fluoresziert. Auf diese Weise kann die Energieübertragung direkt beobachtet werden.  
	
	FRET ist nur möglich, wenn das Emissionsspektrum des Donors und das Absorptionsspektrum des Akzeptors ausreichend überlappen. Die Übertragung beruht auf einer Dipol-Dipol-Wechselwirkung zwischen Donor und Akzeptor. In der Regel erfolgt der Transfer zwischen Singulett-Zuständen (S$_1$ des Donors, S$_0$ des Akzeptors) und wird daher als Singulett-Singulett-Transfer bezeichnet.  
	
	Die Transfereffizienz \(E\) gibt an, welcher Anteil der angeregten Donormoleküle ihre Energie erfolgreich an den Akzeptor überträgt. Da \(E\) stark vom Abstand \(R_{DA}\) abhängt, lassen sich aus den gemessenen Fluoreszenzspektren quantitative Informationen über den Abstand zwischen Donor und Akzeptor gewinnen.	
	{\let\clearpage\relax\chapter{Experimentalteil}}
	\section{Versuchsablauf}
	Die bereitgestellten Proben wurden entsprechend des Skripts vermessen.\cite{Skript} 
	\section{Messergebnisse \& Auswertung}
	\subsection{Korrektur der Spektren}
	Die Korrektur der Spektren wurden durch Abzug der Messwerte der Lösemittelmessungen von jenen der Probelösungen durchgeführt.
	\subsection{Berechnung von $R_0$}
	\begin{equation}
		\mathrm{A_{norm}=\frac{A_{mess}-MIN(A_i)}{MAX(A_i)-MIN(A_i)}}\label{eq:Norm_min_max}
	\end{equation}
	\begin{figure}[H]
		\centering
		\includegraphics[width=0.8\textwidth]{6.1.1a.pdf}
		\caption[Fluoreszenz- \& Anregungsspektren des Lösemittels.]{Fluoreszenz- \& Anregungsspektren des Lösemittels, Messungen für $\mathrm{D_{em}}$ (\textcolor{green}{\rule[0.5ex]{1em}{0.5pt}}),  $\mathrm{D_{ex}}$ (\textcolor{blue}{\rule[0.5ex]{1em}{0.5pt}}), $\mathrm{A_{em}}$ (\textcolor{violet}{\rule[0.5ex]{1em}{0.5pt}}) \& $\mathrm{A_{ex}}$ (\textcolor{black}{\rule[0.5ex]{1em}{0.5pt}}).}
	\end{figure}
	\begin{figure}[H]
		\centering
		\includegraphics[width=0.8\textwidth]{6.1.1b.pdf}
		\caption[Fluoreszenz- \& Anregungsspektren der Proben.]{Fluoreszenz- \& Anregungsspektren der Proben, Messungen für $\mathrm{LF\,D_{em}}$ (\textcolor{green}{\rule[0.5ex]{1em}{0.5pt}}) \&  $\mathrm{HF\,D_{em}}$ (\textcolor{blue}{\rule[0.5ex]{1em}{0.5pt}})).}
	\end{figure}
	\begin{equation}
		f_{D/A}(\lambda)=\frac{F_{D/A}}{\sum F_{D/A}}\label{eq:korr_Fluor_lambda}
	\end{equation}
	\begin{equation}
		\varepsilon(\lambda)_{D/A}=\frac{A_{\mathrm{Alexa488/Cy5}}}{MAX(A_i)}\varepsilon_{\mathrm{Alexa488/Cy5}}\label{eq:korr_Extinktionskoeff}
	\end{equation}
	\begin{equation}
		J=\int_{480}^{660}f_D(\lambda)\varepsilon(\lambda)_A\lambda^4\,d\lambda\label{eq:spektr_Überlappungsint}
	\end{equation}
	\begin{equation}
		R_0=0,2108\sqrt[6]{\kappa^2\Phi_Dn^{-4}J}\label{eq:R0}
	\end{equation}
	\subsection{Verhältnismethode}
	
	\subsection{3D-Modelle}
	\section{Diskussion \& Fehlerbetrachtung}
	
	\chapter{Zusammenfassung}
	
	\listoffigures
	\addcontentsline{toc}{chapter}{Abbildungsverzeichnis}
	\printbibliography
\end{document}
